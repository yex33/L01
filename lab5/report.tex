\documentclass[titlepage, 12pt]{article}

\usepackage{framed}
\usepackage{enumitem}
\usepackage{geometry}
\geometry{
  letterpaper,
  margin=1in,
}

\usepackage{graphicx}
\graphicspath{{./images/}}
\usepackage{float}
\usepackage{subcaption}
\usepackage{amssymb}

\title{SE 2XB3 Group 4 Report 5}
\author{
  Huang, Kehao \\
  400235182 \\
  \texttt{huangk53@mcmaster.ca} \\
  L01
  \and
  Jiao, Anhao \\
  400251837 \\
  \texttt{jiaoa3@mcmaster.ca} \\
  L01
  \and
  Ye, Xunzhou \\
  400268576 \\
  \texttt{yex33@mcmaster.ca} \\
  L01
}
\date{26 February 2021}

\begin{document}
\maketitle{}

\newpage{}

\section{Building Heaps}
\label{sec:build}

\begin{description}
\item[\texttt{build\_heap\_1}] A loose upper bound is easy to establish since
  \texttt{sink} is \(\mathcal{O}(\lg{n})\) and less than \(n\) nodes are
  non-leaf nodes. Therefore, this algorithm takes at most
  \(\mathcal{O}(n\lg{n})\). However, to develop a tight upper bound, notice that
  different \texttt{sink} calls operate on ``mini-heaps'' of different \(n\).
  For example, for the first non-leaf node, the height, or \(\lg{n}\), of the
  heap containing this node and it children is only 1. The complexity of all
  \texttt{sink} operations on a heap with \(n\) non-leaf nodes can be expressed
  as the sum of a series:
  \begin{displaymath}
    \sum_{h = 0}^{\lg{n}} 2^h (\lg{n} - h) = 2n - \lg{n} - 2
  \end{displaymath}
  Each level of height \(h\) has at most \(2^h\) nodes. Sinking each node on the
  \(h\)th level takes at most \(\lg{n} - h\) swaps. Therefore, the tight bound
  of \texttt{build\_heap\_1} is concluded to be \(\mathcal{O}(n)\).
\item[\texttt{build\_heap\_2}] Assume appending a node to the bottom of the heap
  takes the amortized time \(\mathcal{O}(1)\). The complexity of the
  \texttt{insert} operation is the complexity of
  \texttt{bubble\_up}/\texttt{swim}, \(\mathcal{O}(\lg{n})\). Since \(n\) nodes
  would be inserted to build the heap, a loose upper bound of this heap building
  algorithm is \(\mathcal{O}(n\lg{n})\).
\item[\texttt{build\_heap\_3}] One round of calling
  \texttt{sink}/\texttt{heapify} on every node has complexity of
  \(\mathcal{O}(\lg{n})\) \texttt{sink} operations times n nodes,
  \(\mathcal{O}(n\lg{n})\). Each round of \texttt{sink} operations is able to
  move a node up only one level in the heap. In the worst case, for the actual
  root (maximum/minimum) of the heap to travel from the bottom level to the top
  level, \(\lg{n}\) (the height of the heap) rounds of \texttt{sink} operations
  are required. Though there is also a helper function \texttt{is\_heap} in each
  round of the \(n\) \texttt{sink} operations, contributing a \(\mathcal{O}(n)\)
  complexity to \texttt{build\_heap\_3}, it is on a smaller scale compared to
  the complexity of the \texttt{sink} operations. Thus, the expected complexity
  of this heap building algorithm is \(\mathcal{O}(n (\lg{n})^2)\).
\end{description}

\section{\(k\)-Heap}
\label{sec:kheap}

The asymptotic complexity of \texttt{sink} is believed to be \(\mathcal{O}(k
\log_{k}{n})\). In a \(k\)-heap, the nodes are organized in a complete \(k\)-ary
tree of height \(\log_{k}{n}\). In the worst case, a node \(e\), needs to “sink”
through \(\log_{k}{n}\) levels. On each level, \(k\) comparisons are required to
find the maximum element on the level. The maximum would then be swapped with
node \(e\). Hence, the complexity of \texttt{sink} is \(k\) comparisons times
\(\log_{k}{n}\) levels, \(\mathcal{O}(k \log_{k}{n})\).

A \(k\)-heap is essentially a \(k\)-ary tree. The height of a \(k\)-ary tree is
\(\log_{k}{n}\), which is smaller than that of a binary tree (heap), \(\lg{n}\).
The \texttt{swim} operation on a \(k\)-ary heap is \(\mathcal{O}(\log_{k}{n})\).
On the other hand, a larger \(k\) results in requiring more comparisons for
\texttt{sink} on each level of the tree. Therefore, in cases where applications
depending solely on the \texttt{swim} operation are prioritized over all other
operations, \(k\)-heaps of a large \(k\) have a huge advantage over binary
heaps. In other cases where \texttt{sink} or both \texttt{swim} and
\texttt{sink} are heavily used, such as Heapsort, \(k\)-heaps of any \(k\) are
not significantly better than binary heaps. In fact, the function family \(y =
k\log_{k}{x}\) minimizes for \(k \in \mathbb{N}\) at \(k = 3\), which means
3-ary heap is better than binary heap in most aspects, and any \(k\)-heap of \(k
> 3\) trades off its \texttt{sink} performance for that of \texttt{swim}.


\end{document}
